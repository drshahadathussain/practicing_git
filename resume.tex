Shahadat Hussain
RESEARCH ASSOCiATE · DiViSiON OF ENGiNEERiNG · CTP LAB
New York University, Saadiyat Island, Abu Dhabi, UAE 129188
􀄦 +971 50 628 0812 | 􀄇 shahadat_ind@yahoo.com | 􀁵 https://sites.google.com/view/shahadathussain | 􀁝
linkedin.com/in/shahadathussain | 􀁼 shah.bksc

Summary
An adept researcher with over four years of expertise in 3D printing, actively involved in various projects focusing on additive
manufacturing. Proficient in exploring innovative advancements in 3D printing automation, skilled at fabricating
prototypes, utilizing advanced materials characterization techniques, conducting mechanical testing, and trained in
data analysis and visualization using Python libraries like pandas, numpy, scipy, matplotlib and seaborn, and scientific
publications.

Work Experience
Research Associate Abu Dhabi, UAE
Division of Engineering, New York University Sep 2024 ‑ Present
• Hybrid Manufacturing including additive manufacturing via stereolithography, fused deposition method, polyjet
printing, laser powder bed fusion, robotic arm electric arc printing and investment casting.
• Conducting tensile and fatigue testing, utilizing characterization methods including SEM, XRD, EDS, and metallography,
coupled with data analysis and visualization tools like Origin Lab, MS Excel, and various Python libraries.
• Working on review articles on 3D printing applications in civil engineering sectors.

Postdoctoral Researcher Abu Dhabi, UAE
Mechanical Engineering, Khalifa University Feb 2020 ‑ Jun 2024
• Analyzing the attributes of additively manufactured NiTi TPMS structures, such as Schwartz primitive and Schoen
gyroid lattices, categorized as porous, cellular, and architected materials.
• Conducting tensile and fatigue testing, utilizing characterization methods including SEM, XRD, EDS, and metallography,
coupled with data analysis and visualization tools like Origin Lab, MS Excel, and various Python libraries.
• Published 5 research articles in reputed scientific journals and developed lab manuals.

Project Fellow Bhopal, India
CSIR‑Advanced Materials and Processes Research Institute May 2013 ‑ Aug 2014
• Extensive scientific research experience with experimentation, data analysis, teamwork, problem‑solving, and effective
communication. Proficient in relationship building, negotiation skills.
• Engaged in diverse experimental tasks: alloy synthesis, metallography, heat treatment, and material characterization
using spectroscopy, microscopy, diffractometry, calorimetry, hardness, tensile, and hot rolling.
• Published 2 research articles, presented research via posters and oral presentations, collaborated with Indian Institute
of Technology, Madras.

Education
Academy of Scientific and Innovative Research Bhopal, India
PhD (Engineering Sciences) in Materials Science and Technology Aug 2014 ‑ Jun 2019
• Doctoral research on Cu‑Al‑Ni shape memory alloys, studied impact of grain refiners, alloying additions, and processing
parameters on shape memory properties for high‑temperature applications.
• Published 2 research articles, and delivered 5 conference presentations.

University Institute of Technology‑RGPV Bhopal, India
Bachelor of Engineering in Mechanical Engineering Aug 2008 ‑ May 2012
• Graduated with first division aggregate.
• Underwent 2‑week industrial training at Steel Authority of India Limited (SAIL)‑Bokaro.

Skills
Software: Computer‑Aided Design (CAD) such as FreeCAD 0.20, Python programming language and its libraries including
pandas, numpy, scipy, matplotlib, and seaborn for data analysis and data visualization, Xpert HighScore Plus,
ImageJ, Origin Lab, microscope image analysis software like Gatan Suite, and word processing tools like LaTeX.

Fabrication: Additive Manufacturing and Vacuum induction melting technique, Heat treatment of alloys and quenching.
Mechanical Testing: Tensile and compression testing using Instron 5969 static loading machine with Bluehill software
and Fatigue testing using Instron 8872 dynamic loading machine with WaveMatrix software.
Metallography: Grinding, polishing, hot and cold mounting, ultrasonic and plasma cleaning using Struer and Buehler
equipment, and chemical etching of metal alloys.
Materials Characterization: Scanning Electron Microscopes (SEM): JEOL 7610F, FEI Nova NanovSEM, and FEI Quanta
3D FIB, for imaging and EDS analysis, X‑ray Diffraction: Bruker D2 Phasor and Riggaku MiniFlex II, Optical Microscopy:
Leica microscopes, Thermal Analysis: Differential Scanning Calorimetry using Mettler Toledo DSC 1 and Setaram instruments,
and Optical Emission Spectroscopy: Bruker Q4 Tasman.

Projects
Additive manufacturing of NiTi shape memory alloys, constitutive
modeling and fatigue failure criteria Abu Dhabi, UAE
Khalifa University Feb 2020 ‑ Present
• Involved in 3D printing of NiTi samples and TPMS structures
• Extensive characterizations performed using SEM, TEM, XRD, AFM, DSC, and fatigue testing
• Project Outcomes: Made novel findings and published 5 research articles, conference presentations and development
of lab manuals.

Design and Development of Thermo‑Responsive and Magnetic Shape
Memory Materials and Devices for Engineering Applications Bhopal, India
CSIR‑Advanced Materials and Processes Research Institute Jun 2013 ‑ Aug 2014
• Develop copper‑based shape memory materials with superior mechanical properties, high transition temperatures,
and cost‑effective production of shape memory wires and strips.
• Project outcomes: Published 2 research articles in scientific journals, presented at national events, collaborated
with Indian Institute of Technology, Madras. Well‑received oral and poster presentations.

Certifications
• Introduction to Project Management, June 2024, IBM
• Additive Manufacturing Specialization, June 2024, Arizona State University
• Data Science Professional Certificate, April 2024, IBM
• Python for Everybody Specialization, February 2024, University of Michigan
• FreeCAD: A Basic 3D Modeling, June 2023, Udemy

Publications
• Shahadat Hussain, Ali N Alagha, Wael Zaki, (2024) Phase Transformation Behavior of NiTi Triply Periodic Minimal
Surface Lattices Fabricated by Laser Powder Bed Fusion. Journal of Materials Engineering and Performance,
https://doi.org/10.1007/s11665-024-09162-7.
• Shahadat Hussain, Ali N Alagha,Gregory N. Haidemenopoulos, Wael Zaki, (2023) Microstructural and surface
analysis of NiTi TPMS lattice sections fabricated by laser powder bed fusion. Journal of Manufacturing Processes,
102:375‑386, https://doi.org/10.1016/j.jmapro.2023.07.055.
